%-------------------------------------------------------------------------------
%	SECTION TITLE
%-------------------------------------------------------------------------------
%\cvsectionspecifiedcolor{rainbowcolor-indigo}{\faGraduationCap \acvHeaderIconSep Education}
\cvsectionspecifiedcolor{awesome-red}{\faGraduationCap \acvHeaderIconSep Education}


%-------------------------------------------------------------------------------
%	CONTENT
%-------------------------------------------------------------------------------
\begin{cventries}

\vspace{-0.25cm}

%---------------------------------------------------------
  \cventry
    {\href{https://www.ubi.pt/en/course/849}{PhD in Computer Science and Engineering %(18/20 points)
    }} % Degree
    {\href{https://www.di.ubi.pt/}{Department of Computer Science and Engineering}, \href{http://www.ubi.pt/}{Universidade da Beira Interior}} % Institution
    {Covilh\~{a}, Portugal} % Location
    {Jan. 2013 - June 2017} % Date(s)
    {
      \begin{cvitems} % Description(s) bullet points
        \item[] {\textbf{Thesis:} Influence-based Motion Planning Algorithms for Games}
        %\item[] {Areas of Study: Artificial Intelligence, Motion Planning, and Game Engines}
        \item[] {\textbf{Supervisor:} \href{http://www.di.ubi.pt/~agomes/}{Professor Abel Gomes}}
        \item[] {\textbf{Context:} Implementation of a modular game engine, suitable for teaching a video games technologies course.%, and serve as a framework to experiment with scene management for MMOGs. 
        Path finding algorithms merged with influence maps.}
        \item[] {\textbf{Functions:} Survey state-of-the-art regarding (i)modular game engine architectures, (ii)video game technologies teaching methodologies, (iii)influence maps and path finding algorithms. %(iv)scene management in virtual environments. 
        Game engine development (JOT). %Implementation of techniques for state management in MMOGs.         
        %Integration of the proposed solution with a network simulator to test the solution(s) performance.         
        Implementation of a novel pathfinding algorithm, and two novel techniques to integrate influence maps with pathfinders. Writing of journal/conference scientific articles and thesis, presentation of conference articles, and thesis oral presentation/defense.}
        \item[] {\textbf{Technologies:} 
        \textcolor{rainbowcolor-olive}{GIT}, 
        \textcolor{rainbowcolor-olive}{NetBeans}, 
        \textcolor{rainbowcolor-indigo}{Java 1.6-1.8}, 
        \textcolor{rainbowcolor-indigo}{GridGain}, 
        \textcolor{rainbowcolor-indigo}{JGroups}, 
        %\textcolor{rainbowcolor-indigo}{JogAmp}, 
        \textcolor{rainbowcolor-indigo}{OpenGL}, 
        \textcolor{rainbowcolor-indigo}{Apache Maven}, 
        \textcolor{rainbowcolor-indigo}{Apache Math Commons}, 
        %C++, Omnet++, Eclipse, LibreOffice,
        %\textcolor{rainbowcolor-indigo}{LaTeX}, 
        \textcolor{rainbowcolor-orange}{Linux % (Ubuntu, Mint, Fedora, and OpenSuse)
        }, 
        \textcolor{rainbowcolor-orange}{Windos XP-7}
		.}
      \end{cvitems}
    }

\hspace{-0.18cm}    
    
%---------------------------------------------------------
  \cventry
    {\href{https://www.ubi.pt/en/course/64}{Bachelor's Degree in Information Technologies and Systems %(15/20 points)
    }} % Degree
    {} % Institution
    {} % Location
    {Sept. 2010 - July 2011} % Date(s)
    {
      %\begin{cvitems} % Description(s) bullet points
        %\item[] {Final Project at \href{http://www.ubi.pt/}{Instituto de Telecomunica\c{c}~oes}: Seamless zoning algorithms for MMOGs over a Grid (18/20 points)}
        %\item[] {\textbf{Final Project:} Seamless zoning algorithms for MMOGs over a Grid (18/20 points)}
		%\item[] {\textbf{Supervisor:} \href{http://www.di.ubi.pt/~agomes/}{Professor Abel Gomes}}
		%\item[] {\textbf{Context:} Bachelors final project consisting in extending with scene management techniques an existing Java video game, to explore grid computing frameworks and the possibility of converting multiplayer online video games into MMOGs.}
		%\item[] {\textbf{Functions:} Gathering/analysis of state-of-the-art regarding scene management in virtual environments. Evaluation of communication, storage, and computation grid solutions (e.g., GridGain). Implementation of existing techniques into an existing video game. Writing of the project final report and presentation, and oral presentation former.}
		%\item[] {\textbf{Technologies:} Java 1.6, GridGain, Apache Math Commons, JogAmp, NetBeans, LaTeX.}
      %\end{cvitems}      
    }
  \vspace{-0.7cm}  
%---------------------------------------------------------    
  \cventry
    {\href{https://www.ubi.pt/en/course/804}{Master's Degree in Computer Science and Engineering %(18/20 points)
    }} % Degree
    {} % Institution
    {} % Location
    {Sept. 2007 - Oct. 2009} % Date(s)
    {
%      \begin{cvitems} % Description(s) bullet points
%        \item[] {\textbf{Thesis:} Real-Time 3D Rendering of Water using CUDA (19/20 points)}
%        %\item[] {Area of Study: Natural Phenomena Real-Time Simulation}
%        \item[] {\textbf{Supervisor:} \href{http://www.di.ubi.pt/~agomes/}{Professor Abel Gomes}}
%		\item[] {\textbf{Context:} Extending a 2D fluids simulation algorithm to 3D in the GPU resorting to CUDA.}
%		\item[] {\textbf{Functions:} Gathering/analysis of state-of-the-art regarding fluid simulation in virtual environments. Porting a 2D fluids simulation (Jos Stam Stable fluids) algorithm to 3D in the GPU resorting to CUDA. Writing of journal/conference scientific articles and thesis, presentation of conference articles, and thesis oral presentation/defense.}
%        \item[] {\textbf{Technologies:}  
%        \textcolor{rainbowcolor-olive}{Visual Studio 2005-2008},
%        \textcolor{rainbowcolor-indigo}{CUDA 2.0}, 
%        \textcolor{rainbowcolor-indigo}{C/C++}, 
%        %\textcolor{rainbowcolor-indigo}{LaTeX}, 
%        \textcolor{rainbowcolor-orange}{Linux % (Ubuntu, Mint, Fedora, and OpenSuse)
%        }, 
%        \textcolor{rainbowcolor-orange}{Windos XP-7}       
%      	.}
%      \end{cvitems}
    }
	\hspace{-0.18cm}
    \vspace{-0.6cm} 
%---------------------------------------------------------
  \cventry
    {\href{https://www.ubi.pt/en/course/42}{Bachelor's Degree in Computer Science and Engineering %(13/20 points)
    }} % Degree
    {} % Institution
    {} % Location
    {Sept. 2001 - July 2007} % Date(s)
    {
    }
  %\vspace{-0.55cm}
%---------------------------------------------------------    
%  \cventry
%    {High school Professional Degree in Electrotechnology and Electronics} % Degree
%    {\href{http://espeniche.pt/sitesp/index.php}{Secondary/High School of Peniche}} % Institution
%    {Peniche, Portugal} % Location
%    {Sept. 1998 - July 2001} % Date(s)    
%    {
%    }
%---------------------------------------------------------
\end{cventries}
