%-------------------------------------------------------------------------------
%	SECTION TITLE
%-------------------------------------------------------------------------------
%\cvsectionspecifiedcolor{rainbowcolor-green}{\faBriefcase \acvHeaderIconSep Experience}
\cvsectionspecifiedcolor{rainbowcolor-red}{\faBriefcase \acvHeaderIconSep Experience}

%Adaptive – modifying the system to cope with changes in the software environment (DBMS, OS) [4]
%Perfective – implementing new or changed user requirements which concern functional enhancements to the software
%Corrective – diagnosing and fixing errors, possibly ones found by users [4]
%Preventive – increasing software maintainability or reliability to prevent problems in the future [4]


%-------------------------------------------------------------------------------
%	CONTENT
%-------------------------------------------------------------------------------
\begin{cventries}

%---------------------------------------------------------
  \cventry
    {Software Engineer} % Job title
    {\href{https://www.reditpro.com/}{Red IT} \textbf{(Professional)}} % Organization
    {Lisbon, Portugal} % Location
    {August 2017 - Present} % Date(s)
    {
      \begin{cvitems} % Description(s) of tasks/responsibilities      
%        \item {Assistance in the integration of a login functionality managed by Azure AD within a vacation management online solution.}
%%        \item {Development and support of an information management system in the areas of social action and health. %used by a Portuguese collective entity of private law and administrative public utility.} 
%        \item {Development and bug support for an in use information management system in the areas of social action and health. %used by a Portuguese collective entity of private law and administrative public utility.} 
%        \linebreak (\textbf{cannot disclose from which due to confidentiality agreement terms}).}		
		\item {\textbf{Context:} 
Adaptive, perfective and corrective software maintenance for an information management system used in the areas of social action and health %(cannot disclose from which due to confidentiality agreement terms)
, lets refer to it as X.}
		\item {\textbf{Functions:} Extension and maintenance of all Model-View-Controller (MVC) functionalities, including models, views, controllers, and SQL scripts to update and include new functionalities on X database. Functional and aesthetic bug resolution in views, controllers, and models of X.}		
		\item {\textbf{Technologies:} TFS, jQuery, Ajax, JSON, Bootstrap, C\#, MVC 5.0, MS SQL-server 2014, Visual Studio 2013.}
      \end{cvitems}
    } 
    
%---------------------------------------------------------
  \cventry
    {Software Engineer} % Job title
    {} % Organization
    {} % Location
    {July 2017 - August 2017} % Date(s)
    {
      \begin{cvitems} % Description(s) of tasks/responsibilities      
%        \item {Assistance in the integration of a login functionality managed by Azure AD within a vacation management online solution.}
%%        \item {Development and support of an information management system in the areas of social action and health. %used by a Portuguese collective entity of private law and administrative public utility.} 
%        \item {Development and bug support for an in use information management system in the areas of social action and health. %used by a Portuguese collective entity of private law and administrative public utility.} 
%        \linebreak (\textbf{cannot disclose from which due to confidentiality agreement terms}).}		
		\item {\textbf{Context:} Perfective and corrective software maintenance in the integration of a login functionality managed by Azure AD within a near release vacation management online solution, lets refer to it as Y.}
		\item {\textbf{Functions:} Study of Azure AD usage scenarios. Assessment of how could Azure AD be integrated in the at the time login process of Y.}		
		\item {\textbf{Technologies:} Git, Angular2/4, JSON, Bootstrap, Node.js, C\#, MVC 5.0, MS SQL-server 2016, Visual Studio 2017, Visual Studio Code, Azure AD.}
      \end{cvitems}
    }   

%%---------------------------------------------------------
%  \cventry
%    {PhD Researcher for <\href{http://www.fct.pt/index.phtml.en}{Foundation for Science and Technology}>} % Job title
%    {\href{http://www.it.ubi.pt/medialab}{Instituto de Telecomunica\c{c}\~{o}es and Graphics \& Media Laboratory}} % Organization
%    {Covilh\~a, Portugal} % Location
%    {Jan. 2012 - Present} % Date(s)
%    {
%      \begin{cvitems} % Description(s) of tasks/responsibilities
%      %\item {Engineering and design of a novel modular game engine architecture.}
%      \item {Implementation and development of novel pathfinders reactive to influence maps.}%a novel pathfinder and of two novel motion planning algorithms, that integrate influence maps with pathfinders.}
%      %\item {Conducting research and case studies on leading edge algorithms and technologies. Writing scientific papers.}
%      %\item {Lecturing about achieved results and possible new research trends.}
%      \item {Conducting research, writing scientific papers, and lecturing about achieved results and possible new research trends.}
%%        \item {Conducting research and case studies on leading edge algorithms and technologies. Writing scientific papers.}
%%        \item {Engineering and design of modular game engines.}
%%        \item {Implementation and development of innovative motion planning techniques.}
%%        \item {Lecturing about achieved results and possible new research trends.}
%%        \item[] {}
%%        \item[] {\textbf{Notable Accomplishments:}}      
%%        \item {Proposal of a novel modular game engine architecture.}
%%        \item {Proposal of two novel motion planning algorithms, that integrate influence maps with pathfinders.}
%%        \item {Proposal of a novel pathfinder.}
%      \end{cvitems}
%    }  
    
%---------------------------------------------------------
  \cventry
    {Research Associate for <PTDC/EIA/70830/2006, \href{http://www.di.ubi.pt/~agomes/moggy/index.html}{MOGGY - A Browser-Based Massive Multiplayer Online Game Engine Architecture}> Project} % Job title
    {\href{http://www.it.ubi.pt/medialab}{Instituto de Telecomunica\c{c}\~{o}es and Graphics \& Media Laboratory} \textbf{(Academic)}} % Organization
    {} % Location
    {Mar. 2008 - Dec. 2011} % Date(s)
    {
      \begin{cvitems} % Description(s) of tasks/responsibilities
%		%\item {Implementation of a game engine to test state management processing, storage and communication techniques.}        
%		\item {Engineering and design of a novel modular game engine architecture to test state management techniques.}
%        \item {Implementation of Grid-based solutions to transform multi-player games into MMOGs.}
%        %\item {Conducting research and review the state of art regarding Massive Multiplayer Online Games (MMOGs).}		
		\item {\textbf{Context:} Study and ultimately attempt to developed a game engine for MMOGs to work on the browser.}
		\item {\textbf{Functions:} Developed a game engine prototype in Java to test state management algorithms. Assisted in porting an existing multi-player game (Jake2) to a MMOFPS resorting to GridGain. Implemented fluid simulators in the GPU. Elaboration and public presentation of scientific articles.}
		\item {\textbf{Technologies:} Java 1.6, GridGain, Jgroups, JogAmp, Maven, Apache Math Commons, C/C++, CUDA 2.0, LaTeX,  Beamer.} %, TexMaker, JabRef.}
      \end{cvitems}
    }     
    
%---------------------------------------------------------
  \cventry
    {Lab Instructor for the
        11498-Computer Science and Engineering \& \linebreak
		11156-Game Design and Development: 
		\href{http://www.di.ubi.pt/~agomes/tjv/}{Video Games Technologies}	Course 
		\linebreak and the
		5385-Computer Science and Engineering: 		
		\href{http://www.di.ubi.pt/~agomes/cg/}{Computer graphics} Course} % Job title
%    {Lab Instructor for the 
%        \href{http://www.di.ubi.pt/~agomes/tjv/}
%        {Video Games Technologies} course} % Job title
    {\href{http://www.ubi.pt}{Universidade da Beira Interior} \textbf{(Academic)}} % Organization
    {Covilh\~a, Portugal} % Location
    {\linebreak Jan. 2012 - July 2016 \linebreak Jan. 2012 - July 2012} % Date(s)
    {
      \begin{cvitems} % Description(s) of tasks/responsibilities
%		\item {Developed laboratory course material, including practical sheets and tests (available upon request).}
%        %\item {Responsible for 0.5~hour lecture and supervision of 1.5~hour laboratory.} 
%        %\item {Participated in practical project joint assessment with course supervisor.}
%        \item {Responsible for 0.5~hour lecture and supervision of 1.5~hour laboratory, and practical project joint assessment.}% with course supervisor	.}		
		\item {\textbf{Context:} Lab. assistant in practical component of the video game technologies and computer graphics courses.}
		\item {\textbf{Functions:} Developed lab. course material, including practical sheets and tests (available upon request). Responsible for 0.5 hour lecture and supervision of 1.5 hour lab. Participated in practical project joint assessment with course supervisor.}
		\item {\textbf{Technologies:} Java 1.6-1.8, JmonkeyEngine, NetBeans, Whiteboard, LibreOffce, LaTeX, Beamer.} %, TexMaker, JabRef.}
      \end{cvitems}
    }       
    
%---------------------------------------------------------
%  \cventry
%    {Lab Instructor for the
%        11498-Computer Science and Engineering \& \linebreak
%		11156-Game Design and Development: \href{http://www.di.ubi.pt/~agomes/tjv/}
%		{Video Games Technologies} Course} % Job title
%%    {Lab Instructor for the 
%%        \href{http://www.di.ubi.pt/~agomes/tjv/}
%%        {Video Games Technologies} course} % Job title
%    {\href{http://www.ubi.pt}{Universidade da Beira Interior}} % Organization
%    {Covilh\~a, Portugal} % Location
%    {Jan. 2012 - July 2016} % Date(s)
%    {
%%      \begin{cvitems} % Description(s) of tasks/responsibilities
%%        \item {Responsible for 0.5~hour lecture and supervision of
%%                1.5~hour laboratory.} %where graduate students learn 
%%                %technologies, techniques, algorithms, data structures, 
%%                %and mathematics behind the design and development of game engines, 
%%                %instead of games themselves.}           
%%        \item {Participated in practical project joint assessment 
%%				with course supervisor.}
%%		\item {Developed laboratory course material, 
%%                including practical sheets and tests available upon request.}
%%      \end{cvitems}	   
%    }
%    
%%---------------------------------------------------------  
%  \cventry
%    {Lab Instructor for the 5385-Computer Science and Engineering:
%     \href{http://www.di.ubi.pt/~agomes/cg/}{Computer graphics} Course} % Job title
%%    {Lab Instructor for the \href{http://www.di.ubi.pt/~agomes/cg/}{Computer graphics} course} % Job title
%    {} % Organization
%    {} % Location
%    {Jan. 2012 - July 2012} % Date(s)
%    {
%      \begin{cvitems} % Description(s) of tasks/responsibilities
%        \item {Responsible for 0.5~hour lecture and supervision of 
%        		1.5~hour laboratory.} %where undergraduate students are introduced 
%                %2D and 3D computer graphics 
%                %trough GLUT/OpenGL interactive applications programming.}
%        \item {Participated in practical project joint assessment 
%				with course supervisor.}		
%      \end{cvitems}
%    }   
    
%---------------------------------------------------------    
  \cventry    
    {Research Associate for <POCI/V/04.01302/0155/0002/2006, "Metodologias Din\^amicas para o Sucesso em Matem\'atica> Project} % Job title
    {} % Organization
    {} % Location
    {July 2007 - Apr. 2008} % Date(s)
    {
      \begin{cvitems} % Description(s) of tasks/responsibilities
%        \item {Creation of dynamic contents (presentations and work sheets) for theoretical and laboratory mathematics courses.}
%        \item {Integration of dynamic content with a content management system (CMS), assessment and extension of CMSs.}
		%\item {Research, implementation, and testing of novel modern ways to divulge a math department.}
		\item {\textbf{Context:} Develop dynamic ways to dynamize a math department and math teaching at an university level.}
		\item {\textbf{Functions:} Creation of dynamic contents (presentations and work sheets) for theoretical and laboratory mathematics courses, i.e., dynamic presentations intended to make math learning more interactive and available outside the class room.}
  		\item {\textbf{Technologies:} Blackboard, Mathematica, Matlab, LaTeX, Beamer.} %, TexMaker.}
      \end{cvitems}
    }  
     
%---------------------------------------------------------   
\end{cventries}
