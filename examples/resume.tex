%!TEX TS-program = xelatex
%!TEX encoding = UTF-8 Unicode
% Awesome CV LaTeX Template for CV/Resume
%
% This template has been downloaded from:
% https://github.com/posquit0/Awesome-CV
%
% Author:
% Claud D. Park <posquit0.bj@gmail.com>
% http://www.posquit0.com
%
% Template license:
% CC BY-SA 4.0 (https://creativecommons.org/licenses/by-sa/4.0/)
%


%-------------------------------------------------------------------------------
% CONFIGURATIONS
%-------------------------------------------------------------------------------
% A4 paper size by default, use 'letterpaper' for US letter
\documentclass[11pt, a4paper]{awesome-cv}

% Configure page margins with geometry
\geometry{left=1.5cm, top=1.5cm, right=1.5cm, bottom=1.5cm, footskip=.5cm}

% Specify the location of the included fonts
\fontdir[fonts/]

% Color for highlights
% Awesome Colors: awesome-emerald, awesome-skyblue, awesome-red, awesome-pink, awesome-orange
%                 awesome-nephritis, awesome-concrete, awesome-darknight
%\colorlet{awesome}{awesome-red}
% Uncomment if you would like to specify your own color
% \definecolor{awesome}{HTML}{CA63A8}

% Colors for text
% Uncomment if you would like to specify your own color
% \definecolor{darktext}{HTML}{414141}
% \definecolor{text}{HTML}{333333}
% \definecolor{graytext}{HTML}{5D5D5D}
% \definecolor{lighttext}{HTML}{999999}

% Set false if you don't want to highlight section with awesome color
\setbool{acvSectionColorHighlight}{true}

% If you would like to change the social information separator from a pipe (|) to something else
\renewcommand{\acvHeaderSocialSep}{\quad\textbar\quad}

\usepackage{xcolor}

%-------------------------------------------------------------------------------
%	PERSONAL INFORMATION
%	Comment any of the lines below if they are not required
%-------------------------------------------------------------------------------
\photo[circle,noedge,left]{profile.png}
\name{Gon\c{c}alo N.}{Paiva Amador\vspace{0.1cm}}
%\position{Software Engineer {\enskip\cdotp\enskip} Lecturer {\enskip\cdotp\enskip} Project Manager}
\personalinformation{
%\faHome \acvHeaderIconSep Peniche, Portugal {\enskip\cdotp\enskip} 
\faHome \acvHeaderIconSep Lisbon, Portugal {\enskip\cdotp\enskip} 
\faFlag \acvHeaderIconSep Portuguese {\enskip\cdotp\enskip} 
\faBirthdayCake \acvHeaderIconSep 27/07/1983 {\enskip\cdotp\enskip} 
\faMars \acvHeaderIconSep Male {\enskip\cdotp\enskip} 
%\faFemale \acvHeaderIconSep Single\vspace{0.1cm}
\faMale \acvHeaderIconSep Single
\vspace{0.1cm}
}
%\address{Rua das Oper\'arias Conserveiras, n.$^o$ 15, 2520-639, Peniche, Portugal}

%\mobile{(+351) 96-281-68-58} 
\email{g.n.p.amador@gmail.com}
%\homepage{www.posquit0.com}
\github{g-amador}
%\bitbucket{g-amador}
%\stackoverflow{SO-id}{SO-name}
%\researchgate{Goncalo\_Amador} %No icon not working yet
\linkedin{g-amador}
%\twitter{@twit}
%\skype{gnpa\_skype}
%\reddit{reddit-id}
%\extrainfo{extra informations} 

\quote{"First, solve the problem. Then, write the code." John Johnson}

%-------------------------------------------------------------------------------
\begin{document}

% Print the header with above personal informations
% Give optional argument to change alignment(C: center, L: left, R: right)
\makecvheader[L]

% Print the footer with 3 arguments(<left>, <center>, <right>)
% Leave any of these blank if they are not needed
\makecvfooter
  {\today}
  {Gon\c{c}alo N. Paiva Amador~~~·~~~Résumé}
  {\thepage}


%-------------------------------------------------------------------------------
%	CV/RESUME CONTENT
%	Each section is imported separately, open each file in turn to modify content
%-------------------------------------------------------------------------------
%-------------------------------------------------------------------------------
%	SECTION TITLE
%-------------------------------------------------------------------------------
\cvsectionspecifiedcolor{rainbowcolor-red}{\faUser \acvHeaderIconSep Summary}


%-------------------------------------------------------------------------------
%	CONTENT
%-------------------------------------------------------------------------------
\begin{cvsummary}

%---------------------------------------------------------
Software Developer with 6 months of experience.
Former Project Researcher, Laboratory Professor, and Scientific Presenter with over 9 years of experience.
Highly motivated, communicative, self-sufficient, and versatile professional with solid academic background in Computer Science and Engineering, namely in: 
%Proved ability to address problems with a tendentiously unbiased and professional view, namely regarding: 
Game Engine Technologies, Teaching methodologies, HPC, Geometric Computing, and HCI.
Known as a team player %, multi-tasker, 
and constant self-driven learner; 
striving to address novel and exciting challenges. %by integrating, adapting, developing novel knowledge/tools.
Preference to work with and/or manage teams in order to grow personally and professionally.
\end{cvsummary}

%-------------------------------------------------------------------------------
%	SECTION TITLE
%-------------------------------------------------------------------------------
%\cvsectionspecifiedcolor{rainbowcolor-orange}{\faGears \acvHeaderIconSep Research Interests}
\cvsectionspecifiedcolor{awesome-red}{\faGears \acvHeaderIconSep Areas of Interest}


%-------------------------------------------------------------------------------
%	CONTENT
%-------------------------------------------------------------------------------
\begin{cventries}

%---------------------------------------------------------

  \cventry
    {} 
    {}
    {}
    {}
    {    	
	  \vspace{-1.7cm}
      \begin{multicols}{2}
      \begin{cvitems}
        \item {Touch/Voice/Camera-based HCI technologies.}
        \item {3D Animation/Modelling \& Geometrical Computing.}
        \item {Multi-Core CPU/GPU and Cloud computing.}
        \item {Computational Fluids Dynamics (CFD).}
        \item {Artificial Intelligence, Robotics, and Cybernetics.}
        \item {Computer Games \& Gamification.}
        \item {Workforce scheduling \& management software.}
        \item {Network Security \& Configuration \& Administration.}
      \end{cvitems}
	  \end{multicols}
	  \vspace{-0.5cm}
    }

%---------------------------------------------------------
\end{cventries}

%-------------------------------------------------------------------------------
%	SECTION TITLE
%-------------------------------------------------------------------------------
%\cvsectionspecifiedcolor{rainbowcolor-yellow}{\faBarChart \acvHeaderIconSep Skills}
\cvsectionspecifiedcolor{awesome-skyblue}{\faBarChart \acvHeaderIconSep Skills}


%-------------------------------------------------------------------------------
%	CONTENT
%-------------------------------------------------------------------------------

\begin{cventries}

%---------------------------------------------------------
  \vspace{0.5cm}
  \cventry
    {Communication} % title
    {} 
    {} 
    {} 
    {
      \vspace{0.4cm}
      \begin{cvitems} % Description(s)
        \item {Good communication skills.}
        \item {Event promotions.}
        \item {Customer service experience.} 
        \item {Ability to work in team acquired in several collaborative projects.}         
      \end{cvitems}
    }  
%  %\vspace{-0.4cm} 
    
%---------------------------------------------------------
  \cventry
    {Organization} % title
    {} 
    {} 
    {} 
    {
      \vspace{0.4cm} 
      \begin{cvitems} % Description(s)
        \item {Good independent work, acquired while working as a trainee.} 
        \item {Experience organizing and managing events: festivals, concerts, parties, workshops.}
      \end{cvitems}
    }        
  %\vspace{-0.4cm}   
    

%---------------------------------------------------------
  \cventry
    {Work/Design} % title
    {} 
    {} 
    {} 
    {
      \vspace{0.4cm}
      \begin{cvitems} % Description(s)
        \item {Project methodology.}
        \item {Rapid construction of models and test models.}
        \item {Processing of statistical data.} 
      \end{cvitems}
    }  
  \vspace{-0.2cm} 


%---------------------------------------------------------
  \cventry
    {Technologies \& Programming} % title
    {} 
    {} 
    {} 
    {
      \begin{cvitems} % Description(s)
        \item[]{AutoCAD 2D/3D, CADWorx, Navisworks, Solidworks, Rhinoceros 3D, Autodesk 2D, Autodesk Inventor, CAD/CAM}
        \item[]{Rendering 3D models}
  		\item[]{Microsoft Office, CorelDraw}
		\item[]{Arduino}
		\item[]{MailChimp Plataform}
		\item[]{C++}
      \end{cvitems}
    }  
  \vspace{0.2cm} 

%---------------------------------------------------------

\end{cventries}
%-------------------------------------------------------------------------------
%	SECTION TITLE
%-------------------------------------------------------------------------------
%\cvsectionspecifiedcolor{rainbowcolor-green}{\faBriefcase \acvHeaderIconSep Experience}
\cvsectionspecifiedcolor{rainbowcolor-red}{\faBriefcase \acvHeaderIconSep Experience}

%Adaptive – modifying the system to cope with changes in the software environment (DBMS, OS) [4]
%Perfective – implementing new or changed user requirements which concern functional enhancements to the software
%Corrective – diagnosing and fixing errors, possibly ones found by users [4]
%Preventive – increasing software maintainability or reliability to prevent problems in the future [4]


%-------------------------------------------------------------------------------
%	CONTENT
%-------------------------------------------------------------------------------
\begin{cventries}

%---------------------------------------------------------
  \cventry
    {Software and Oracle Database Developer} % Job title
    {\href{https://www.amaris.com/}{Amaris} \textbf{(Professional)}} % Organization
    {Lisbon, Portugal} % Location
    {Jan. 2019 - Present} % Date(s)
    {
      \begin{cvitems} % Description(s) of tasks/responsibilities      
		\item {\textbf{Context:} account migration among banks, adaptive, perfective and corrective banking software development. 
}
		\item {\textbf{Functions:} inspecting PL-SQL queries, the sequence of queries, and then fine tuning those queries prior to sending data. Day to day support activities, investigate and address issues regarding quality of sent data with the business area. Adaptive, perfective and corrective banking software development.}		
		\item {\textbf{Technologies:} Oracle SQL developer, PL-SQL, Redmine, Skype/Skype for Business, MS Office Suite, Apache AirFlow, Apache Ant, Apache Maven, Apache Tomcat, Apache Struts, Eclipse, Java 6-8.}		
      \end{cvitems}
    } 
    
%---------------------------------------------------------
  \cventry
    {Technical Account Manager} % Job title
    {} % Organization
    {} % Location
    {Feb. 2018 - Dec. 2018} % Date(s)
    {
      \begin{cvitems} % Description(s) of tasks/responsibilities      
		\item {\textbf{Context:} Primary technical contact point with one or more clients, assisting in planning, debugging, and supervising ongoing critical business applications.}
		\item {\textbf{Functions:} provide technical support for customers to support pre-sales and post-sales processes, address all product-related queries on time, create learning materials and train customers to use products effectively, assist in the creation of support documentation for existing software/service products, provide developers with customer's feedback to help identify potential new features or products, report on product performance, identify solutions to reduce support costs, analyze customer's needs and suggest upgrades or additional features to meet their requirements.}		
		\item {\textbf{Technologies:} Azure DevOps, TFS, IIS 6-8, VNC, mRemote, Cisco AnyConnect, %Cisco VPN client, 
FortiClient, Teamviewer, Skype/Skype for Business, WebEx, MS Office Suite, PowerBI, SSRS/SSIS/SSAS, Postman, SoapUI, T-SQL, MS SQL-server 2012-2017, Zendesk.}		
      \end{cvitems}
    } 
    
%---------------------------------------------------------
  \cventry
    {Software Consultant} % Job title
    {\href{https://www.reditpro.com/}{RedIT} \textbf{(Professional)}} % Organization
    {Lisbon, Portugal} % Location
    {Oct. 2017 - Jan. 2018} % Date(s)
    {
      \begin{cvitems} % Description(s) of tasks/responsibilities      
		\item {\textbf{Context:} 
Adaptive software development and integration with Model-Based architecture for an energy sector company %(cannot disclose from which due to confidentiality agreement terms)
.}
		\item {\textbf{Functions:} Integration of Uber-like functionalities into existing Model-Based architecture modules. PL-SQL database replication. Integration, algorithm development, and prototyping of route calculation and schedule assigning software into existing Model-Based architecture modules.}		
		\item {\textbf{Technologies:} JavaScript, Git, C\#, MS SQL-server 2016, Visual Studio 2017, Oracle SQL developer, T-SQL, PL-SQL, Slack.}
      \end{cvitems}
    } 
    
%---------------------------------------------------------
  \cventry
    {Software Consultant} % Job title
    {} % Organization
    {} % Location
    {Aug. 2017 - Sept. 2018} % Date(s)
    {
      \begin{cvitems} % Description(s) of tasks/responsibilities      
%        \item {Assistance in the integration of a login functionality managed by Azure AD within a vacation management online solution.}
%%        \item {Development and support of an information management system in the areas of social action and health. %used by a Portuguese collective entity of private law and administrative public utility.} 
%        \item {Development and bug suTFS, Account Management, teacherpport for an in use information management system in the areas of social action and health. %used by a Portuguese collective entity of private law and administrative public utility.} 
%        \linebreak (\textbf{cannot disclose from which due to confidentiality agreement terms}).}		
		\item {\textbf{Context:} 
Adaptive, perfective and corrective software maintenance for an information management system used in the areas of social action and health %(cannot disclose from which due to confidentiality agreement terms)
, lets refer to it as X.}
		\item {\textbf{Functions:} Extension and maintenance of all Model-View-Controller (MVC) functionalities, including models, views, controllers, and SQL scripts to update and include new functionalities on X database. Functional and aesthetic bug resolution in views, controllers, and models of X.}		
		\item {\textbf{Technologies:} TFS, jQuery, Ajax, JSON, Bootstrap, C\#, MVC 5.0, MS SQL-server 2014, Visual Studio 2013.}
      \end{cvitems}
    } 
    
%---------------------------------------------------------
  \cventry
    {Software Consultant} % Job title
    {} % Organization
    {} % Location
    {July 2017} % Date(s)
    {
      \begin{cvitems} % Description(s) of tasks/responsibilities  
		\item {\textbf{Context:} Perfective and corrective software maintenance in the integration of a login functionality managed by Azure AD within a near release vacation management online solution, lets refer to it as Y.}
		\item {\textbf{Functions:} Study of Azure AD usage scenarios. Assessment on how to integrate (at the time) Y login process with Azure AD.}		
		\item {\textbf{Technologies:} Git, Angular2/4, JSON, Bootstrap, Node.js, C\#, MVC 5.0, MS SQL-server 2016, Visual Studio 2017, Visual Studio Code, Azure AD.}
      \end{cvitems}
    }   

%%---------------------------------------------------------
%  \cventry
%    {PhD Researcher for <\href{http://www.fct.pt/index.phtml.en}{Foundation for Science and Technology}>} % Job title
%    {\href{http://www.it.ubi.pt/medialab}{Instituto de Telecomunica\c{c}\~{o}es and Graphics \& Media Laboratory}} % Organization
%    {Covilh\~a, Portugal} % Location
%    {Jan. 2012 - Present} % Date(s)
%    {
%      \begin{cvitems} % Description(s) of tasks/responsibilities
%      %\item {Engineering and design of a novel modular game engine architecture.}
%      \item {Implementation and development of novel pathfinders reactive to influence maps.}%a novel pathfinder and of two novel motion planning algorithms, that integrate influence maps with pathfinders.}
%      %\item {Conducting research and case studies on leading edge algorithms and technologies. Writing scientific papers.}
%      %\item {Lecturing about achieved results and possible new research trends.}
%      \item {Conducting research, writing scientific papers, and lecturing about achieved results and possible new research trends.}
%%        \item {Conducting research and case studies on leading edge algorithms and technologies. Writing scientific papers.}
%%        \item {Engineering and design of modular game engines.}
%%        \item {Implementation and development of innovative motion planning techniques.}
%%        \item {Lecturing about achieved results and possible new research trends.}
%%        \item[] {}
%%        \item[] {\textbf{Notable Accomplishments:}}      
%%        \item {Proposal of a novel modular game engine architecture.}
%%        \item {Proposal of two novel motion planning algorithms, that integrate influence maps with pathfinders.}
%%        \item {Proposal of a novel pathfinder.}
%      \end{cvitems}
%    }  
    
%---------------------------------------------------------
  \cventry
    {Research Associate for <PTDC/EIA/70830/2006, \href{http://www.di.ubi.pt/~agomes/moggy/index.html}{MOGGY - A Browser-Based Massive Multiplayer Online Game Engine Architecture}> Project} % Job title
    {\href{http://www.it.ubi.pt/medialab}{Instituto de Telecomunica\c{c}\~{o}es and Graphics \& Media Laboratory} \textbf{(Academic)}} % Organization
    {} % Location
    {Mar. 2008 - Dec. 2011} % Date(s)
    {
      \begin{cvitems} % Description(s) of tasks/responsibilities
%		%\item {Implementation of a game engine to test state management processing, storage and communication techniques.}        
%		\item {Engineering and design of a novel modular game engine architecture to test state management techniques.}
%        \item {Implementation of Grid-based solutions to transform multi-player games into MMOGs.}
%        %\item {Conducting research and review the state of art regarding Massive Multiplayer Online Games (MMOGs).}		
		\item {\textbf{Context:} Study and ultimately attempt to Develop a game engine for MMOGs to work on the browser.}
		\item {\textbf{Functions:} Develop a Java game engine prototype, to test state management algorithms. Assist in porting an existing multi-player game (Jake2) to a MMOFPS resorting to GridGain. Implement fluid simulators in the GPU. Elaboration and public presentation of scientific conference articles.}
		\item {\textbf{Technologies:} Java 1.6, GridGain, Jgroups, JogAmp, Apache Math Commons, C/C++, CUDA 2.0, LaTeX,  Beamer.} %, TexMaker, JabRef.}
      \end{cvitems}
    }     
    
%---------------------------------------------------------
  \cventry
    {Lab Instructor for the
        11498-Computer Science and Engineering \& \linebreak
		11156-Game Design and Development: 
		\href{http://www.di.ubi.pt/~agomes/tjv/}{Video Games Technologies}	Course 
		\linebreak and the
		5385-Computer Science and Engineering: 		
		\href{http://www.di.ubi.pt/~agomes/cg/}{Computer graphics} Course} % Job title
%    {Lab Instructor for the 
%        \href{http://www.di.ubi.pt/~agomes/tjv/}
%        {Video Games Technologies} course} % Job title
    {\href{http://www.ubi.pt}{Universidade da Beira Interior} \textbf{(Academic)}} % Organization
    {Covilh\~a, Portugal} % Location
    {\linebreak Jan. 2012 - July 2016 \linebreak Jan. 2012 - July 2012} % Date(s)
    {
      \begin{cvitems} % Description(s) of tasks/responsibilities
%		\item {Develop laboratory course material, including practical sheets and tests (available upon request).}
%        %\item {Responsible for 0.5~hour lecture and supervision of 1.5~hour laboratory.} 
%        %\item {Participated in practical project joint assessment with course supervisor.}
%        \item {Responsible for 0.5~hour lecture and supervision of 1.5~hour laboratory, and practical project joint assessment.}% with course supervisor	.}		
		\item {\textbf{Context:} Lab. assistant in practical component of the video game technologies and computer graphics courses.}
		\item {\textbf{Functions:} Develop lab. course material, including practical sheets and tests (available upon request). Responsible for 0.5 hour lecture and supervision of 1.5 hour lab. Participated in practical project joint assessment with course supervisor.}
		\item {\textbf{Technologies:} Java 1.6-1.8, JmonkeyEngine, NetBeans, Whiteboard, LibreOffce, LaTeX, Beamer.} %, TexMaker, JabRef.}
      \end{cvitems}
    }       
    
%---------------------------------------------------------
%  \cventry
%    {Lab Instructor for the
%        11498-Computer Science and Engineering \& \linebreak
%		11156-Game Design and Development: \href{http://www.di.ubi.pt/~agomes/tjv/}
%		{Video Games Technologies} Course} % Job title
%%    {Lab Instructor for the 
%%        \href{http://www.di.ubi.pt/~agomes/tjv/}
%%        {Video Games Technologies} course} % Job title
%    {\href{http://www.ubi.pt}{Universidade da Beira Interior}} % Organization
%    {Covilh\~a, Portugal} % Location
%    {Jan. 2012 - July 2016} % Date(s)
%    {
%%      \begin{cvitems} % Description(s) of tasks/responsibilities
%%        \item {Responsible for 0.5~hour lecture and supervision of
%%                1.5~hour laboratory.} %where graduate students learn 
%%                %technologies, techniques, algorithms, data structures, 
%%                %and mathematics behind the design and development of game engines, 
%%                %instead of games themselves.}           
%%        \item {Participated in practical project joint assessment 
%%				with course supervisor.}
%%		\item {Develop laboratory course material, 
%%                including practical sheets and tests available upon request.}
%%      \end{cvitems}	   
%    }
%    
%%---------------------------------------------------------  
%  \cventry
%    {Lab Instructor for the 5385-Computer Science and Engineering:
%     \href{http://www.di.ubi.pt/~agomes/cg/}{Computer graphics} Course} % Job title
%%    {Lab Instructor for the \href{http://www.di.ubi.pt/~agomes/cg/}{Computer graphics} course} % Job title
%    {} % Organization
%    {} % Location
%    {Jan. 2012 - July 2012} % Date(s)
%    {
%      \begin{cvitems} % Description(s) of tasks/responsibilities
%        \item {Responsible for 0.5~hour lecture and supervision of 
%        		1.5~hour laboratory.} %where undergraduate students are introduced 
%                %2D and 3D computer graphics 
%                %trough GLUT/OpenGL interactive applications programming.}
%        \item {Participated in practical project joint assessment 
%				with course supervisor.}		
%      \end{cvitems}
%    }   
    
%---------------------------------------------------------    
  \cventry    
    {Research Associate for <POCI/V/04.01302/0155/0002/2006, "Metodologias Din\^amicas para o Sucesso em Matem\'atica> Project} % Job title
    {} % Organization
    {} % Location
    {July 2007 - Apr. 2008} % Date(s)
    {
      \begin{cvitems} % Description(s) of tasks/responsibilities
%        \item {Creation of dynamic contents (presentations and work sheets) for theoretical and laboratory mathematics courses.}
%        \item {Integration of dynamic content with a content management system (CMS), assessment and extension of CMSs.}
		%\item {Research, implementation, and testing of novel modern ways to divulge a math department.}
		\item {\textbf{Context:} Develop dynamic ways to dynamize a math department and math teaching at an university level.}
		\item {\textbf{Functions:} Creation of dynamic contents (presentations and work sheets) for theoretical and laboratory mathematics courses, i.e., dynamic presentations intended to make math learning more interactive and available outside the class room.}
  		\item {\textbf{Technologies:} Blackboard, Mathematica, Matlab, LaTeX, Beamer.} %, TexMaker.}
      \end{cvitems}
    }  
     
%---------------------------------------------------------   
\end{cventries}

%-------------------------------------------------------------------------------
%	SECTION TITLE
%-------------------------------------------------------------------------------
%\cvsectionspecifiedcolor{rainbowcolor-blue}{\faCertificate \acvHeaderIconSep Certifications}
\cvsectionspecifiedcolor{awesome-red}{\faCertificate \acvHeaderIconSep Certifications}

%-------------------------------------------------------------------------------
%	CONTENT
%-------------------------------------------------------------------------------
\begin{cventries}

\vspace{-0.25cm}

%---------------------------------------------------------
%  \cventry
%    {SAP Intelligent Robotic Process Automation in a Nutshell (24.0/30)} % Certificate designation
%    {\href{https://www.sap.com/index.html}{SAP}} % Certifacion entity
%    {Online} % Location
%    %{} % Location
%    %{Sept 17, 2018 - Oct 16, 2018} % Date(s)      
%    {Sept. 2019 - Oct. 2019} % Date(s)          
%    {
%	    \begin{cvitems} % Description(s) of tasks/responsibilities  
%        \item[] {Online Certificate: \url{https://open.sap.com/verify/xolav-kibal-bifaf-kyfid-rysek}} %Certificate URL 
%        \item[] {Valid from July 1, 2020 - Present} % Validity
%      \end{cvitems}
%    }
%    \vspace{-0.1cm}
%---------------------------------------------------------
  \cventry
    {Build Mobile Applications with SAP Cloud Platform Mobile Services (227.8/300)} % Certificate designation
    %{\href{https://www.sap.com/index.html}{SAP}} % Certifacion entity
    {} % Certifacion entity
    %{Online} % Location
    {} % Location
    %{Aug 20, 2018 - Oct 02, 2018} % Date(s)      
    {Aug. 2019 - Oct. 2019} % Date(s)          
    {
	    \begin{cvitems} % Description(s) of tasks/responsibilities  
        \item[] {Online Certificate: \url{https://open.sap.com/verify/xicav-sakyr-fopyh-kyfal-tunah}} %Certificate URL 
        \item[] {Valid from October 2, 2019 - Present} % Validity
      \end{cvitems}
    }
    \vspace{-0.1cm}
%---------------------------------------------------------
  \cventry
    {Object-Oriented Programming in Java (188.0/200)} % Certificate designation
    %{\href{https://www.sap.com/index.html}{SAP}} % Certifacion entity
    {} % Certifacion entity
    %{Online} % Location
    {} % Location
    %{June 13, 2018 - July 26, 2018} % Date(s)      
    {June 2018 - July 2018} % Date(s)          
    {
	    \begin{cvitems} % Description(s) of tasks/responsibilities  
        \item[] {Online Certificate: \url{https://open.sap.com/verify/xocis-tyvip-gynuh-byneg-punel}} %Certificate URL 
        \item[] {Valid from July 25, 2018 - Present} % Validity
      \end{cvitems}
    }   
    \vspace{-0.1cm}
%---------------------------------------------------------
  \cventry
    {Cloud-Native Development with SAP Cloud Platform* (249.1/360)} % Certificate designation
    %{\href{https://www.sap.com/index.html}{SAP}} % Certifacion entity
    {} % Certifacion entity
    %{Online} % Location
    {} % Location
    %{Apr. 10, 2018 - May 30, 2018} % Date(s)      
    {Apr. 2018 - May 2018} % Date(s)          
    {
	    \begin{cvitems} % Description(s) of tasks/responsibilities     
        \item[] {Online Certificate: \url{https://open.sap.com/verify/xosen-nykel-vybup-susig-sihim}} %Certificate URL 
        \item[] {Valid from May 30, 2018 - Present} % Validity
      \end{cvitems}
    }   
    \vspace{-0.1cm}
%---------------------------------------------------------
%  \cventry
%    {Design Your First App with Build* (230.0/240)} % Certificate designation
%    %{\href{https://www.sap.com/index.html}{SAP}} % Certifacion entity
%    {} % Certifacion entity
%    %{Online} % Location
%    {} % Location
%    %{Oct. 4, 2017 - Nov. 9, 2017} % Date(s)      
%    {Oct. 2017 - Nov. 2017} % Date(s)      
%    {
%	    \begin{cvitems} % Description(s) of tasks/responsibilities    
%        \item[] {Online Certificate: \url{https://open.sap.com/verify/xoban-setyz-vogaf-leryz-lufor}} %Certificate URL 
%        \item[] {Valid from Nov. 9, 2017 - Present} % Validity
%      \end{cvitems}
%    }      
%    \vspace{-0.1cm}
%---------------------------------------------------------
%  \cventry
%    {Be Visual! Sketching Basics for IT Business (70.7/80)} % Certificate designation
%    %{\href{https://www.sap.com/index.html}{SAP}} % Certifacion entity
%    {} % Certifacion entity
%    %{Online} % Location
%    {} % Location
%    %{Sep. 6, 2017 - Oct. 12, 2017} % Date(s)      
%    {Sep. 2017 - Oct. 2017} % Date(s)          
%    {
%	    \begin{cvitems} % Description(s) of tasks/responsibilities   
%        \item[] {Online Certificate: \url{https://open.sap.com/verify/xuhos-fogot-hytyc-pybin-helen}} %Certificate URL 
%        \item[] {Valid from Oct. 12, 2017 - Present} % Validity
%      \end{cvitems}
%    }      
%    \vspace{-0.1cm}
%---------------------------------------------------------
  \cventry
    {Developing Java-Based Apps on SAP Cloud Platform* (234.3/300)} % Certificate designation
    %{\href{https://www.sap.com/index.html}{SAP}} % Certifacion entity
    {} % Certifacion entity
    %{Online} % Location
    {} % Location
    %{April 5, 2017 - May 18, 2017} % Date(s)      
    {Apr. 2017 - May 2017} % Date(s)      
    {
	    \begin{cvitems} % Description(s) of tasks/responsibilities    
        \item[] {Online Certificate: \url{https://open.sap.com/verify/xuhes-rimus-bikon-roper-hicom}} %Certificate URL 
        \item[] {Valid from May 18, 2017 - Present} % Validity
      \end{cvitems}
    }    
    \vspace{-0.1cm}
%---------------------------------------------------------
%  \cventry
%    {SAP HANA Cloud Platform* Essentials (300/360)} % Certificate designation
%    %{\href{https://www.sap.com/index.html}{SAP}} % Certifacion entity
%    {} % Certifacion entity
%    %{Online} % Location
%    {} % Location
%    %{Feb. 7, 2017 - March 29, 2017} % Date(s)      
%    {Feb. 2017 - Mar. 2017} % Date(s)      
%    {
%	    \begin{cvitems} % Description(s) of tasks/responsibilities   
%        \item[] {Online Certificate: \url{https://open.sap.com/verify/xulor-nynoz-soped-muvuv-zitek}} %Certificate URL 
%        \item[] {Valid from March 29, 2017 - Present} % Validity
%      \end{cvitems}
%    }        
%    \vspace{-0.1cm}
%---------------------------------------------------------  
%  \cventry
%    {Extending SAP S/4HANA with SAP HANA Cloud Platform* (254.0/360)} % Certificate designation
%    %{\href{https://www.sap.com/index.html}{SAP}} % Certifacion entity
%    {} % Certifacion entity
%    %{Online} % Location
%    {} % Location
%    %{Jan. 11, 2017 - March 2, 2017} % Date(s)      
%    {Jan. 2017 - Mar. 2017} % Date(s)      
%    {
%	    \begin{cvitems} % Description(s) of tasks/responsibilities  
%        \item[] {Online Certificate: \url{https://open.sap.com/verify/xugal-vaseb-firih-zadyd-deniv}} %Certificate URL 
%        \item[] {Valid from Feb. 28, 2017 - Present} % Validity
%      \end{cvitems}
%    }
%--------------------------------------------------------- 
  \cventry
    {CCNA Routing and Switching: Introduction to networks} % Certificate designation
    {\href{https://www.cisco.com/}{Cisco}} % Certifacion entity     
    {Covilh\~a, Portugal} % Location
    {Oct. 2013 - Jan. 2014} % Date(s)       
    {
	    \begin{cvitems} % Description(s) of tasks/responsibilities  
        \item[] {Valid from Jan. 2014 - Present} % Validity
      \end{cvitems}
    }

%---------------------------------------------------------
\end{cventries}
%-------------------------------------------------------------------------------
%	SECTION TITLE
%-------------------------------------------------------------------------------
%\cvsectionspecifiedcolor{rainbowcolor-indigo}{\faGraduationCap \acvHeaderIconSep Education}
\cvsectionspecifiedcolor{awesome-red}{\faGraduationCap \acvHeaderIconSep Education}


%-------------------------------------------------------------------------------
%	CONTENT
%-------------------------------------------------------------------------------
\begin{cventries}

\vspace{-0.25cm}

%---------------------------------------------------------
  \cventry
    {\href{https://www.ubi.pt/en/course/849}{PhD in Computer Science and Engineering %(18/20 points)
    }} % Degree
    {\href{https://www.di.ubi.pt/}{Department of Computer Science and Engineering}, \href{http://www.ubi.pt/}{Universidade da Beira Interior}} % Institution
    {Covilh\~{a}, Portugal} % Location
    {Jan. 2013 - June 2017} % Date(s)
    {
      \begin{cvitems} % Description(s) bullet points
        \item[] {\textbf{Thesis:} Influence-based Motion Planning Algorithms for Games}
        %\item[] {Areas of Study: Artificial Intelligence, Motion Planning, and Game Engines}
        \item[] {\textbf{Supervisor:} \href{http://www.di.ubi.pt/~agomes/}{Professor Abel Gomes}}
        \item[] {\textbf{Context:} Implementation of a modular game engine, suitable for teaching a video games technologies course.%, and serve as a framework to experiment with scene management for MMOGs. 
        Path finding algorithms merged with influence maps.}
        \item[] {\textbf{Functions:} Survey state-of-the-art regarding (i)modular game engine architectures, (ii)video game technologies teaching methodologies, (iii)influence maps and path finding algorithms. %(iv)scene management in virtual environments. 
        Game engine development (JOT). %Implementation of techniques for state management in MMOGs.         
        %Integration of the proposed solution with a network simulator to test the solution(s) performance.         
        Implementation of a novel pathfinding algorithm, and two novel techniques to integrate influence maps with pathfinders. Writing of journal/conference scientific articles and thesis, presentation of conference articles, and thesis oral presentation/defense.}
        \item[] {\textbf{Technologies:} 
        \textcolor{rainbowcolor-olive}{GIT}, 
        \textcolor{rainbowcolor-olive}{NetBeans}, 
        \textcolor{rainbowcolor-indigo}{Java 1.6-1.8}, 
        \textcolor{rainbowcolor-indigo}{GridGain}, 
        \textcolor{rainbowcolor-indigo}{JGroups}, 
        %\textcolor{rainbowcolor-indigo}{JogAmp}, 
        \textcolor{rainbowcolor-indigo}{OpenGL}, 
        \textcolor{rainbowcolor-indigo}{Apache Maven}, 
        \textcolor{rainbowcolor-indigo}{Apache Math Commons}, 
        %C++, Omnet++, Eclipse, LibreOffice,
        %\textcolor{rainbowcolor-indigo}{LaTeX}, 
        \textcolor{rainbowcolor-orange}{Linux % (Ubuntu, Mint, Fedora, and OpenSuse)
        }, 
        \textcolor{rainbowcolor-orange}{Windos XP-7}
		.}
      \end{cvitems}
    }

\hspace{-0.18cm}    
    
%---------------------------------------------------------
  \cventry
    {\href{https://www.ubi.pt/en/course/64}{Bachelor's Degree in Information Technologies and Systems %(15/20 points)
    }} % Degree
    {} % Institution
    {} % Location
    {Sept. 2010 - July 2011} % Date(s)
    {
      %\begin{cvitems} % Description(s) bullet points
        %\item[] {Final Project at \href{http://www.ubi.pt/}{Instituto de Telecomunica\c{c}~oes}: Seamless zoning algorithms for MMOGs over a Grid (18/20 points)}
        %\item[] {\textbf{Final Project:} Seamless zoning algorithms for MMOGs over a Grid (18/20 points)}
		%\item[] {\textbf{Supervisor:} \href{http://www.di.ubi.pt/~agomes/}{Professor Abel Gomes}}
		%\item[] {\textbf{Context:} Bachelors final project consisting in extending with scene management techniques an existing Java video game, to explore grid computing frameworks and the possibility of converting multiplayer online video games into MMOGs.}
		%\item[] {\textbf{Functions:} Gathering/analysis of state-of-the-art regarding scene management in virtual environments. Evaluation of communication, storage, and computation grid solutions (e.g., GridGain). Implementation of existing techniques into an existing video game. Writing of the project final report and presentation, and oral presentation former.}
		%\item[] {\textbf{Technologies:} Java 1.6, GridGain, Apache Math Commons, JogAmp, NetBeans, LaTeX.}
      %\end{cvitems}      
    }
  \vspace{-0.7cm}  
%---------------------------------------------------------    
  \cventry
    {\href{https://www.ubi.pt/en/course/804}{Master's Degree in Computer Science and Engineering %(18/20 points)
    }} % Degree
    {} % Institution
    {} % Location
    {Sept. 2007 - Oct. 2009} % Date(s)
    {
%      \begin{cvitems} % Description(s) bullet points
%        \item[] {\textbf{Thesis:} Real-Time 3D Rendering of Water using CUDA (19/20 points)}
%        %\item[] {Area of Study: Natural Phenomena Real-Time Simulation}
%        \item[] {\textbf{Supervisor:} \href{http://www.di.ubi.pt/~agomes/}{Professor Abel Gomes}}
%		\item[] {\textbf{Context:} Extending a 2D fluids simulation algorithm to 3D in the GPU resorting to CUDA.}
%		\item[] {\textbf{Functions:} Gathering/analysis of state-of-the-art regarding fluid simulation in virtual environments. Porting a 2D fluids simulation (Jos Stam Stable fluids) algorithm to 3D in the GPU resorting to CUDA. Writing of journal/conference scientific articles and thesis, presentation of conference articles, and thesis oral presentation/defense.}
%        \item[] {\textbf{Technologies:}  
%        \textcolor{rainbowcolor-olive}{Visual Studio 2005-2008},
%        \textcolor{rainbowcolor-indigo}{CUDA 2.0}, 
%        \textcolor{rainbowcolor-indigo}{C/C++}, 
%        %\textcolor{rainbowcolor-indigo}{LaTeX}, 
%        \textcolor{rainbowcolor-orange}{Linux % (Ubuntu, Mint, Fedora, and OpenSuse)
%        }, 
%        \textcolor{rainbowcolor-orange}{Windos XP-7}       
%      	.}
%      \end{cvitems}
    }
	\hspace{-0.18cm}
    \vspace{-0.6cm} 
%---------------------------------------------------------
  \cventry
    {\href{https://www.ubi.pt/en/course/42}{Bachelor's Degree in Computer Science and Engineering %(13/20 points)
    }} % Degree
    {} % Institution
    {} % Location
    {Sept. 2001 - July 2007} % Date(s)
    {
    }
  %\vspace{-0.55cm}
%---------------------------------------------------------    
%  \cventry
%    {High school Professional Degree in Electrotechnology and Electronics} % Degree
%    {\href{http://espeniche.pt/sitesp/index.php}{Secondary/High School of Peniche}} % Institution
%    {Peniche, Portugal} % Location
%    {Sept. 1998 - July 2001} % Date(s)    
%    {
%    }
%---------------------------------------------------------
\end{cventries}

%-------------------------------------------------------------------------------
%	SECTION TITLE
%-------------------------------------------------------------------------------
\cvsectionspecifiedcolor{awesome-nephritis}{\faGlobe \acvHeaderIconSep Domínio de Línguas}


%-------------------------------------------------------------------------------
%	CONTENT
%-------------------------------------------------------------------------------
\begin{cventries}

%---------------------------------------------------------
  \cventry
    {}
    {}
    {}
    {} 
    {
	  \vspace{-0.5cm}
      \begin{cvitems} % Description(s) of experience/contributions/knowledge
        \item[] {\hspace{-.31cm}\textbf{Utilizador Avançado (CEFRL:C2):} Português (Nativo).}
        \item[] {\hspace{-.31cm}\textbf{Utilizador Avançado (CEFRL:C1):} Inglês.}
        \item[] {\hspace{-.31cm}\textbf{Utilizador Independente (CEFRL:B1):} Espanhol, Italiano e Francês.}
      \end{cvitems}
    }

%---------------------------------------------------------
\end{cventries}

%-------------------------------------------------------------------------------
%	SECTION TITLE
%-------------------------------------------------------------------------------
%\cvsectionspecifiedcolor{rainbowcolor-cyan}{\faInfoCircle \acvHeaderIconSep Additional Information}
\cvsectionspecifiedcolor{rainbowcolor-red}{\faInfoCircle \acvHeaderIconSep Additional Information}


%-------------------------------------------------------------------------------
%	CONTENT
%-------------------------------------------------------------------------------
\begin{cventries}

%---------------------------------------------------------
  \cventry
    {} 
    {} 
    {} 
    {} 
    {
      \vspace{-.6cm}
      \begin{cvitems} % Description(s) of experience/contributions/knowledge
        \item[] {\hspace{-.4cm} \textbf{Researcher} with 9 international scientific articles published (1 journal and 8 conferences).}
        \item[] {\hspace{-.4cm} \textbf{Keynote speaker} at 8 technical, technological and scientific events.}
%        \item[] {\hspace{-.4cm} \textbf{Lab Instructor} with proven ability to create course materials and evaluate practical projects.}
        \item[] {\hspace{-.4cm} \textbf{B1/B Drivers license}}
      \end{cvitems}
    }

%---------------------------------------------------------
\end{cventries}

%%-------------------------------------------------------------------------------
%	SECTION TITLE
%-------------------------------------------------------------------------------
\cvsection{Extracurricular Activity}


%-------------------------------------------------------------------------------
%	CONTENT
%-------------------------------------------------------------------------------
\begin{cventries}

%---------------------------------------------------------
  \cventry
    {Core Member \& President at 2013} % Affiliation/role
    {PoApper (Developers' Network of POSTECH)} % Organization/group
    {Pohang, S.Korea} % Location
    {Jun. 2010 - PRESENT} % Date(s)
    {
      \begin{cvitems} % Description(s) of experience/contributions/knowledge
        \item {Reformed the society focusing on software engineering and building network on and off campus.}
        \item {Proposed various marketing and network activities to raise awareness.}
      \end{cvitems}
    }

%---------------------------------------------------------
  \cventry
    {Member} % Affiliation/role
    {PLUS (Laboratory for UNIX Security in POSTECH)} % Organization/group
    {Pohang, S.Korea} % Location
    {Sep. 2010 - Oct. 2011} % Date(s)
    {
      \begin{cvitems} % Description(s) of experience/contributions/knowledge
        \item {Gained expertise in hacking \& security areas, especially about internal of operating system based on UNIX and several exploit techniques.}
        \item {Participated on several hacking competition and won a good award.}
        \item {Conducted periodic security checks on overall IT system as a member of POSTECH CERT.}
        \item {Conducted penetration testing commissioned by national agency and corporation.}
      \end{cvitems}
    }

%---------------------------------------------------------
  \cventry
    {Member} % Affiliation/role
    {MSSA (Management Strategy Club of POSTECH)} % Organization/group
    {Pohang, S.Korea} % Location
    {Sep. 2013 - PRESENT} % Date(s)
    {
      \begin{cvitems} % Description(s) of experience/contributions/knowledge
        \item {Gained knowledge about several business field like Management, Strategy, Financial and marketing from group study.}
        \item {Gained expertise in business strategy areas and inisght for various industry from weekly industry analysis session.}
      \end{cvitems}
    }

%---------------------------------------------------------
\end{cventries}

%%-------------------------------------------------------------------------------
%	SECTION TITLE
%-------------------------------------------------------------------------------
\cvsection{Honors \& Awards}


%-------------------------------------------------------------------------------
%	SUBSECTION TITLE
%-------------------------------------------------------------------------------
\cvsubsection{International}


%-------------------------------------------------------------------------------
%	CONTENT
%-------------------------------------------------------------------------------
\begin{cvhonors}

%---------------------------------------------------------
  \cvhonor
    {Finalist} % Award
    {DEFCON 22nd CTF Hacking Competition World Final} % Event
    {Las Vegas, U.S.A} % Location
    {2014} % Date(s)

%---------------------------------------------------------
  \cvhonor
    {Finalist} % Award
    {DEFCON 21st CTF Hacking Competition World Final} % Event
    {Las Vegas, U.S.A} % Location
    {2013} % Date(s)

%---------------------------------------------------------
  \cvhonor
    {Finalist} % Award
    {DEFCON 19th CTF Hacking Competition World Final} % Event
    {Las Vegas, U.S.A} % Location
    {2011} % Date(s)

%---------------------------------------------------------
  \cvhonor
    {6th Place} % Award
    {SECUINSIDE Hacking Competition World Final} % Event
    {Seoul, S.Korea} % Location
    {2012} % Date(s)

%---------------------------------------------------------
\end{cvhonors}


%-------------------------------------------------------------------------------
%	SUBSECTION TITLE
%-------------------------------------------------------------------------------
\cvsubsection{Domestic}


%-------------------------------------------------------------------------------
%	CONTENT
%-------------------------------------------------------------------------------
\begin{cvhonors}

%---------------------------------------------------------
  \cvhonor
    {3rd Place} % Award
    {WITHCON Hacking Competition Final} % Event
    {Seoul, S.Korea} % Location
    {2015} % Date(s)

%---------------------------------------------------------
  \cvhonor
    {Silver Prize} % Award
    {KISA HDCON Hacking Competition Final} % Event
    {Seoul, S.Korea} % Location
    {2013} % Date(s)

%---------------------------------------------------------
  \cvhonor
    {2nd Award} % Award
    {HUST Hacking Festival} % Event
    {S.Korea} % Location
    {2013} % Date(s)

%---------------------------------------------------------
  \cvhonor
    {3rd Award} % Award
    {HUST Hacking Festival} % Event
    {S.Korea} % Location
    {2010} % Date(s)

%---------------------------------------------------------
  \cvhonor
    {3rd Award} % Award
    {Holyshield 3rd Hacking Festival} % Event
    {S.Korea} % Location
    {2012} % Date(s)

%---------------------------------------------------------
  \cvhonor
    {2nd Award} % Award
    {Holyshield 3rd Hacking Festival} % Event
    {S.Korea} % Location
    {2011} % Date(s)

%---------------------------------------------------------
  \cvhonor
    {5th Place} % Award
    {PADOCON Hacking Competition Final} % Event
    {Seoul, S.Korea} % Location
    {2011} % Date(s)

%---------------------------------------------------------
\end{cvhonors}

%%-------------------------------------------------------------------------------
%	SECTION TITLE
%-------------------------------------------------------------------------------
\cvsection{Presentation}


%-------------------------------------------------------------------------------
%	CONTENT
%-------------------------------------------------------------------------------
\begin{cventries}

%---------------------------------------------------------
  \cventry
    {Presenter for <DEFCON 20th : The way to go to Las Vegas>} % Role
    {6th CodeEngn (Reverse Engineering Conference)} % Event
    {Seoul, S.Korea} % Location
    {Jul. 2012} % Date(s)
    {
      \begin{cvitems} % Description(s)
        \item {Introduced CTF(Capture the Flag) hacking competition and advanced techniques and strategy for CTF}
      \end{cvitems}
    }

%---------------------------------------------------------
\end{cventries}

%%-------------------------------------------------------------------------------
%	SECTION TITLE
%-------------------------------------------------------------------------------
\cvsection{Writing}


%-------------------------------------------------------------------------------
%	CONTENT
%-------------------------------------------------------------------------------
\begin{cventries}

%---------------------------------------------------------
  \cventry
    {Founder \& Writer} % Role
    {A Guide for Developers in Start-up} % Title
    {Facebook Page} % Location
    {Jan. 2015 - PRESENT} % Date(s)
    {
      \begin{cvitems} % Description(s)
        \item {Drafted daily news for developers in Korea about IT technologies, issues about start-up.}
      \end{cvitems}
    }

%---------------------------------------------------------
  \cventry
    {Undergraduate Student Reporter} % Role
    {AhnLab} % Title
    {S.Korea} % Location
    {Oct. 2012 - Jul. 2013} % Date(s)
    {
      \begin{cvitems} % Description(s)
        \item {Drafted reports about IT trends and Security issues on AhnLab Company magazine.}
      \end{cvitems}
    }

%---------------------------------------------------------
\end{cventries}

%%-------------------------------------------------------------------------------
%	SECTION TITLE
%-------------------------------------------------------------------------------
\cvsection{Program Committees}


%-------------------------------------------------------------------------------
%	CONTENT
%-------------------------------------------------------------------------------
\begin{cvhonors}

%---------------------------------------------------------
  \cvhonor
    {Organizer \& Co-director} % Position
    {1st POSTECH Hackathon} % Committee
    {S.Korea} % Location
    {2013} % Date(s)

%---------------------------------------------------------
  \cvhonor
    {Staff} % Position
    {7th Hacking Camp} % Committee
    {S.Korea} % Location
    {2012} % Date(s)

%---------------------------------------------------------
  \cvhonor
    {Problem Writer} % Position
    {1st Hoseo University Teenager Hacking Competition} % Committee
    {S.Korea} % Location
    {2012} % Date(s)

%---------------------------------------------------------
  \cvhonor
    {Staff \& Problem Writer} % Position
    {JFF(Just for Fun) Hacking Competition} % Committee
    {S.Korea} % Location
    {2012} % Date(s)

%---------------------------------------------------------
\end{cvhonors}



%-------------------------------------------------------------------------------
\end{document}
